%
%   INTORDUCTION
%
A project description has to be submitted in order to be admitted to the PhD program in Science at The Faculty of Science and Technology. 
 
The plan is expected to be detailed at a level where possible bottlenecks and risks of not being able to accomplish tasks can be identified. Where such risks are present, the plan must outline alternative plans to secure completion of the PhD degree within the timeframe. 
 
The project description serves multiple purposes. 

\begin{itemize}
    \item It  ensures  that  the  project  is  well  planned  and  that  the  proposed  research  is  manageable within  the  project  period.  Hence,  planning  ahead  and  submitting  a  well  written  project description increases the prospect of completing the PhD studies on time.
    \item For  some  PhD  students,  the  topic  of  the  project  may  not  be  well  known.  Reading  and writing  the  background,  establishing  the  scientific  objectives  and  investigating  which methods to use, is a good way of getting started with the project. 
    \item Writing project proposals is an important activity for researchers and scientists in order to fund their research. Therefore the project description can be considered an integral part of the PhD training, i.e. it is kind of an introduction on how to write a project proposal. 
\end{itemize}

 
The project description shall be written by the PhD student and supervisors together and it must be signed by the PhD student, all supervisors and the head of the department where the student 
will have his/her main affiliation. 
 
The thesis can be written as a collection of articles/manuscripts or as a monograph. It should be clear from the project description what the plan is.

\vspace{1cm}

\begin{tabularx}{\textwidth}{|X|}
    \hline
    \textcolor{darkblue}{PROJECT TITLE} \\
    \hline
    \ProjectTitle \\
    \hline
\end{tabularx}